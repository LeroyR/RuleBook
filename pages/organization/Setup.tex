\chapter{Setup and Preparation}
\label{chap:setup_and_preparation}
Prior to the RoboCup@Home competition, all arriving teams will have an opportunity to set up their robots and prepare for the competition in a \iterm{Setup \& Preparation} phase. This phase is scheduled to start on the first day of the competition, namely when the venue opens and the teams arrive. During the \SetupDays, teams can assemble and test their robots. On the last setup day, a \WelcomeReception{} will be held. To foster the knowledge exchange between teams a conference-like \PS{} takes place during the reception. Additionally, all teams have to get their robots inspected by members of the TC to be allowed to participate in the competition.

\paragraph{Regular tests are not conducted during the setup \& preparation phase.} The competition starts with \SONE{} (see \Rulebook).

\begin{table}[h]
  \newcolumntype{C}[1]{>{\centering\let\newline\\\arraybackslash\hspace{0pt}}m{#1}}
  \newcolumntype{S}{C{1.6cm}}
  \newcolumntype{M}{C{3.2cm}}
  \begin{center}
    \caption{Stage System and Schedule per League (distribution of tests and stages over days may vary)}
    \begin{tabularx}{14.56cm}{S|S|S|S|S|S|S|S}
      \hline
      \multicolumn{2}{|M|}{ \cellcolor[HTML]{FFFFC7}Setup \& \newline Preparation} &
      \multicolumn{2}{M|}{ \cellcolor[HTML]{67FD9A}\iterm{Stage~I}} &
      \multicolumn{2}{M|}{ \cellcolor[HTML]{9698ED}\iterm{Stage~II}} &
      \multicolumn{2}{M|}{ \cellcolor[HTML]{FFCCC9}\iterm{Finals}}\\
      \hline
      %Second row
      \multicolumn{1}{S|}{} &
      \multicolumn{2}{M|}{$\xrightarrow{advance}$\newline All teams that \newline passed Inspection} &
      \multicolumn{2}{M|}{$\xrightarrow{advance}$\newline Best 10 ($<6$) \newline or best 50\% ($\geq 12$)} &
      \multicolumn{2}{M|}{$\xrightarrow{advance}$\newline Best 2 \newline teams} &
      \multicolumn{1}{C{1.2cm}}{~}
      \\ \cline{2-7}
    \end{tabularx}
  \end{center}
\end{table}


\section{General Setup}
\label{sec:general_setup}
Depending on the schedule, the \iterm{Setup \& Preparation} phase lasts for one or two days.

\begin{enumerate}
	\item \textbf{Start:} The \iterm{Setup \& Preparation} starts when the venue opens for the first time.
	\item \textbf{Intention:} During the \iterm{Setup \& Preparation}, teams arrive, bring or receive their robots, and assemble and test them.
	\item \textbf{Tables:} The local organization will set up and randomly assign team tables.
	\item \textbf{Groups:} Depending on the number of teams, the \OC{} may form multiple groups of teams (usually two) for the first (and second stage). The OC will assign teams to groups and announce the assignment to the teams.
	\item \textbf{\Arena{}:} The \Arena{} is available to all teams during the \iterm{Setup \& Preparation}. The OC may schedule special test or mapping slots in which \Arena{} access is limited to one or more teams exclusively (all teams get slots). Note, however, that the \Arena{} may not yet be complete and that the last work is conducted in the \Arena{} during the \SetupDays.
	\item \textbf{Objects:} The delegation of EC, TC, OC and local organizers will buy the objects (see \Rulebook). Note, however, that the objects may not be available at all times and not from the beginning of the \iterm{Setup \& Preparation}.
\end{enumerate}

\section{Welcome Reception}
\label{sec:welcome_reception}
Since Eindhoven 2013, RoboCup@Home holds an own \WelcomeReception{} in addition to the official opening ceremony. During the \WelcomeReception, a \PS{} is held in which teams present their research focus and latest results (see~\refsec{sec:poster_teaser_session}).
\begin{enumerate}
	\item \textbf{Time:} The \WelcomeReception{} is held in the evening of the last setup day.
	\item \textbf{Place:} The \WelcomeReception{} takes place in the @Home \Arena{} and/or in the \AtHome{} team area.
	\item \textbf{Snacks \& drinks:} During the \WelcomeReception{}, snacks and beverages (beers, sodas, etc.) are served.
	\item \textbf{Organization:} It is the responsibility of the OC and the local organizers to organize the \WelcomeReception{} and \PS{}, including:
		\begin{enumerate}
			\item organizing poster stands (one per team) or alternatives for presenting the posters,
			\item organizing snacks and drinks, and
			\item inviting officials, sponsors, the local organization, and the trustees of the RoboCup Federation to the event.
		\end{enumerate}
	\item \textbf{Poster presentation:} During the \WelcomeReception, the teams give a poster presentation on their research focus, recent results, and their scientific contribution.
	Both the poster and the teaser talk are evaluated by a jury (see~\refsec{sec:poster_teaser_session}).
\end{enumerate}

\section{Poster Teaser Session}
\label{sec:poster_teaser_session}
Before the \WelcomeReception{} and \PS, a \iterm{Poster Teaser Session} is held. In this teaser session, each team can give a short presentation of their research and the poster being presented at the poster session.

\subsection{Poster teaser session}
\begin{enumerate}
	\item \textbf{Presentation:} Each team has a maximum of three minutes to give a short presentation of their poster.
	\item \textbf{Time:} The \iterm{Poster Teaser Session} is to be held before the \WelcomeReception{} and \PS{} (see~\refsec{sec:welcome_reception}).
	\item \textbf{Place:} The \PS{} may be held in or around the \Arena{}, but should not interfere with the \RobotInspection{} (see~\refsec{sec:robot_inspection}).
	\item \textbf{Evaluation:} The teaser and poster presentations are evaluated by a jury consisting of members of the other teams. Each team has to provide one person (preferably the team-leader) to follow and evaluate
	%the entire poster teaser session and the poster session. Not providing a person results in no score for this team in the \iterm{Open Challenge}.
	the entire \iterm{Poster Teaser Session} and the \PS.

	%%%%%%%%%%%%%%%%%%%%%%%%%%%%%%%%%%%%%%%%%%%%%%%%%%%%%%%%%%%%%%%%%%%%%%%%%%%%%%
	%
	% In previous years, scores from teaser session has not been used for scoring
	% during competition. Therefore, this section has been commented out
	%
	%%%%%%%%%%%%%%%%%%%%%%%%%%%%%%%%%%%%%%%%%%%%%%%%%%%%%%%%%%%%%%%%%%%%%%%%%%%%%%
	\item \textbf{Criteria:} For each of the following evaluation criteria, a maximum of 10 points is given per jury member:
	\begin{enumerate}
		\item Novelty and scientific contribution
		\item Relevance for RoboCup@Home
		\item Presentation (quality of poster, teaser talk, and discussion during the \PS)
	\end{enumerate}
	\item \textbf{Score:} The points given by each jury member are scaled to obtain a maximum of 50 points. The total score for each team is the mean of the jury member scores. To neglect outliers, the N best and worst scores are left out:
	$$
	score=\frac{\sum \text{team-leader-score}}{\text{number-of-teams}-\left ( 2N+1  \right )},N=\left\{\begin{matrix}
	1, & \text{number-of-teams} \geq 10\\
	2, & \text{number-of-teams} < 10
	\end{matrix}\right.
	$$
	\item \textbf{Sheet collection:} The evaluation sheets are collected by the OC at a later time (announced beforehand by the OC), allowing teams to continue knowledge exchange during the first days of the competition (\SONE).
	\item \textbf{OC Instructions:}
	\begin{itemize}
		\item Prepare and distribute evaluation sheets before the \iterm{Poster Teaser Session}.
		\item Collect the evaluation sheets.
		\item Organize and manage the poster teaser presentations and the \PS.
	\end{itemize}
\end{enumerate}

\section{Robot Inspection}
\label{sec:robot_inspection}
Safety is the most important issue when interacting with humans and operating in the same physical workspace. Because of this, all participating robots are inspected before participating in RoboCup@Home. Every team needs to get their robot(s) inspected and approved so that they can participate in the competition.

\begin{enumerate}
	\item \textbf{Procedure:} The \RobotInspection{} is conducted like a regular test, namely it starts with opening of the arena door (see~\refsec{rule:start_signal}). One team after another (and one robot after another) has to enter the \Arena{} through a designated entrance door, move to the \textit{Inspection Point}, and leave the arena through the designated exit door. In between entering and leaving, the robot is inspected by the \TC.
	\item \textbf{Checked aspects:} During the \RobotInspection{}, each robot is checked for compliance with the competition rules (see~\refsec{rule:robots}), in particular:
	\begin{itemize}
		\item emergency button(s)
		\item collision avoidance (a TC member steps in front of the robot)
		\item voice of the robot (it must be loud and clear)
		\item custom containers (bowl, tray, etc.)
		\item external devices (including wireless network), if any
		\item Alternative human-robot interfaces (see \Rulebook).
		\item \textbi{Standard Platform robots}
		\begin{itemize}
			\item no modifications have been made
			\item specification of the \iaterm{Official Standard Laptop}{OSL} (if required)
		\end{itemize}
		\item \textbi{Open Platform robots}
		\begin{itemize}
			\item robot speed and dimension
			\item start button (if the team requires it)
			\item robot speaker system (plug for RF transmission)
			\item other safety issues (duct tape, hanging cables, sharp edges etc.)
		\end{itemize}
	\end{itemize}
	\item \textbf{Re-inspection:} If the robot is not approved in the inspection, it is the responsibility of the team to get the approval at a later point. Robots are not allowed to participate in any test before passing the \RobotInspection.
	\item \textbf{Time limit:} The robot inspection is interrupted after three minutes (per robot). When told so by the TC --- in case of time interrupt or failure --- the team has to move the robot out of the \Arena{} through the designated exit door.
	\item \textbf{Appearance Evaluation:} In addition to the inspection, the TC evaluates the appearance of the robots. Robots are expected to look nice (no duct tape, no cables hanging loose etc.). In case of objection, the TC may penalize the team with a penalty of maximum 50 points.
	\item \textbf{Accompanying team member:} Each robot is accompanied by only one team member (the team leader is advised).
	\item \textbf{OC instructions (at least two hours before the \RobotInspection):}
	\begin{itemize}
		\item Announce the entry and exit doors.
		\item Announce the location of the \textit{Inspection Point} in the \Arena{}.
		\item Specify and announce where and when the poster teaser and the poster presentation session take place.
	\end{itemize}
\end{enumerate}


% Local Variables:
% TeX-master: "Rulebook"
% End:
