%% %%%%%%%%%%%%%%%%%%%%%%%%%%%%%%%%%%%%%%%%%%%%%%%%%%%%%%%%%%%%%%%%%%%%%%%%%%%
%%
%%          $Id: about.tex 
%%    author(s): RoboCupAtHome Technical Committee(s)
%%  description: About for the RoboCupAtHome organization guidelines
%%
%% %%%%%%%%%%%%%%%%%%%%%%%%%%%%%%%%%%%%%%%%%%%%%%%%%%%%%%%%%%%%%%%%%%%%%%%%%%%

\section*{About this document}
These are the organization guidelines of the RoboCup@Home competition \YEAR. It contains the procedures surrounding the Robocup@Home competition. Both those leading up to the event and those in place during the competition. This document excludes the tests the robot are expected to perform and the scores associated with them. Those can be found in the \Rulebook.
This document has been written by the \YEAR ~RoboCup@Home Technical Committee.

\section*{How to cite this document}
If you refer to RoboCup@Home and this document in particular, please cite:\\

\noindent Justin Hart, Alexander Moriarty, Katarzyna Pasternak, Johannes Kummert,
Alina Hawkin, Vanessa Hassouna, Juan Diego Pena Narvaez, Leroy Ruegemer,
Leander von Seelstrang, Peter Van Dooren, Juan Jose Garcia, Akinobu Mitzutani,
Yuqian Jiang, Tatsuya Matsushima, Riccardo Polvara
\enquote{Robocup@Home \YEAR: Competition organization document,}
\url{https://github.com/RoboCupAtHome/RuleBook/releases/download/2024.1/organization.pdf}, \YEAR.

\begin{center}
	\begin{minipage}{0.8\textwidth}
		\footnotesize%
		\verbatiminput{citation_organization.bib}
	\end{minipage}
\end{center}

\pagebreak