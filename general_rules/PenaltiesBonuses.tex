%%%%%%%%%%%%%%%%%%%%%%%%%%%%%%%%%%%%%%%%%%%%%%%%%%%%%%%%%
\newcommand{\penaltybig}{500~}
\newcommand{\penaltysmall}{250~}

\section{Special penalties and bonuses}\label{sec:special_awards}

\subsection{Penalty for not attending}\label{rule:not_attending}
\begin{enumerate}
	\item \textbf{Automatic schedule:} All teams are automatically scheduled for all tests.

	\item \textbf{Announcement:} If a team cannot participate in a test (for any reason), the team leader has to announce this to the OC at least \timing{60 minutes} before the test slot begins.

	\item \textbf{Penalties:} A team that is not present at the start position when their scheduled test starts, the team is not allowed to participate in the test anymore.
	If the team has not announced that it is not going to participate, it gets a penalty of \scoring{\penaltysmall points}.
\end{enumerate}

\subsection{Extraordinary penalties}\label{rule:extraordinary_penalties}
\begin{enumerate}
	\item \textbf{Penalty for cheating:} If a team member is found cheating or breaking Fair Play, the team will be automatically disqualified of the running test, and a penalty of \scoring{\penaltybig points} is handed out.
	The \iaterm{Technical Committee}{TC} may also disqualify the team for the entire competition.

	\item \textbf{Penalty for faking robots:} If a team starts a test, but it does not solve any of the partial tasks (and is obviously not trying to do so), a penalty of \scoring{\penaltysmall points} is handed out.
	The decision is made by the referees and the monitoring TC member.

	\item \textbf{Extra penalty for collision:} In case of major, (grossly) negligent collisions the \iaterm{Technical Committee}{TC} may disqualify the team for a test (the team receives \scoring{0 points}), or for the entire competition.

	\item \textbf{Not showing up as referee or jury member:} If a team does not provide a referee or jury member (being at the \Arena{} on time), the team receives a penalty of \scoring{\penaltysmall points}, and will be remembered for qualification decisions in future competitions.\\
	Jury members missing a performance to evaluate are excluded from the jury, and the team is disqualified from the test (receives \scoring{0 points}).

	\item \textbf{Modifying or altering standard platform robots:} If any unauthorized modification is found on a Standard Platform League robot, the responsible team will be immediately disqualified for the entire competition while also receiving a penalty of \scoring{\penaltybig points} in the overall score. This behavior will be remembered for qualification decisions in future competitions.\\
\end{enumerate}

\subsection{Bonus for outstanding performance}\label{rule:outstanding_performance}
\begin{enumerate}
	\item For every regular test in \iterm{Stage~I} and \iterm{Stage~II}, the @Home \iaterm{Technical Committee}{TC} can decide to give an extra bonus for \iterm{outstanding performance} of up to 10\% of the maximum test score.

	\item This is to reward teams that do more than what is needed to solely score points in a test but show innovative and general approaches to enhance the scope of @Home.

	\item If a team thinks that it deserves this bonus, it should announce (and briefly explain) this to the \iaterm{Technical Committee}{TC} beforehand.

	\item It is the decision of the \iaterm{Technical Committee}{TC} if (and to which degree) the bonus score is granted.
\end{enumerate}

\subsection{Bonus for perceived social intelligence}\label{rule:perceived_intelligence}
\begin{enumerate}	
    \item For the test \iterm{Receptionist} in \iterm{Stage~I} and, \iterm{Restaurant} in \iterm{Stage~II} tests. Teams are evaluated on the robot's perceived social intelligence 
     performance
    
    \item This bonus, ranging from 0 to 50, depends on the robot's social performance which will be assessed by Referees using a specially designed scale in a survey.

    \item The survey evaluates how the robot recognizes, adapts to, and predicts emotions, behaviors, and cognitions. Each component is assessed through several questions, scored on a scale from 1 to 5, where 1 represents "strongly disagree" and 5 represents "strongly agree." The questions used for evaluation include:
    \begin{itemize}
        \item This robot notices human presence
        \item This robot enjoys meeting people
        \item This robot recognizes individual people
        \item This robot notices when people do things
        \item This robot adapts effectively to different things people do
        \item This robot anticipates people's behavior
        \item This robot tries to be helpful
        \item This robot is trustworthy
        \item This robot cares about others
        \item This robot tries to hurt people
        \item This robot knows if someone is part of a social group
        \item This robot thinks it is better than everyone else
        \item This robot adapts its behavior based upon what people around it know
        \item This robot is impolite
        \item This robot is socially competent
    \end{itemize}
    Detailed information can be found in this \href{https://ipip.ori.org/PSIManualSeptember2018.pdf}{manual}.
    
    \item In each evaluation, the score for every question is summed up and then normalized as follows:
    \[ x' = N \cdot \frac{{x - x_{\text{min}}}}{{x_{\text{max}} - x_{\text{min}}}} \]

    Here, \( N \) is the normalizing factor (50), \( x \) is the sum of scores for each question in the current evaluation, \( x_{\text{min}} \) is set to 15 (since there are 15 questions), and \( x_{\text{max}} \) is set to 75 (as the maximum score for each question is 5).
    
    \item The final score is calculated as the average of individual evaluator scores.
      
    \item The referee has the authority to skip the social assessment test if they believe the robot's performance is not suitable for measurement.
    
    \item After the test is completed, the evaluators will fill out the form, and the scores will be automatically recorded in this \href{https://urjc-my.sharepoint.com/:x:/g/personal/juan_pena_urjc_es/Eb4WyP1H4-FCqeW-qVZ3mF0BRzBR1Cn-J5ltluy90fhTJQ?rtime=guRqYF483Eg}{spreadsheet}.
\end{enumerate}

% Local Variables:
% TeX-master: "../Rulebook"
% End:
